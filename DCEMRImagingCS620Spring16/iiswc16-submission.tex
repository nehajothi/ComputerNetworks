\documentclass[conference]{IEEEtran}

% for numbered citations
\usepackage{cite}
% for figures based on pdfs
\usepackage[pdftex]{graphicx}
 

\begin{document}
\title{Analyzing Parallel Programming Models for Medical Imaging}


% author names and affiliations
% use a multiple column layout for up to three different
% affiliations
\author{\IEEEauthorblockN{Julio C\'ardenas}
\IEEEauthorblockA{Cancer Center\\
University of Arizona\\
Tucson, Arizona\\
Email: http://www.cardenaslab.org}
\and
\IEEEauthorblockN{Forest Danford, Eric Welch, Michelle Mills Strout}
\IEEEauthorblockA{Computer Science\\
University of Arizona\\
Tucson, Arizona\\
Email: \{fdanford, welche, mstrout\}@email.arizona.edu}}

\maketitle

%%%%%%%%%%%%%%%%%%%%%%%%%%
\begin{abstract}
Medical image analysis can help identify tumors and determine how well such
tumors might respond to treatment.
Scientists exploring new techniques for performing such analyses typically
prototype their algorithms in Matlab only to find that the lack of performance
significantly constrains how much of the captured images can be analyzed,
the size of the images, and the precision of the analysis.
In this paper, we present a framework in Matlab for comparing different implementations
of a medical imaging technique both in terms of performance but
also how much the technique is impacted by noise in the picture.
We use this framework to compare implementations of a medical imaging
technique implemented in Parallel Matlab, Julia, ...
We find ...
\end{abstract}

\IEEEpeerreviewmaketitle


%%%%%%%%%%%%%%%%%%%%%%%%%%
\section{Introduction}

~\cite{Luszczek2009}

%%%%%%%%%%%%%%%%%%%%%%%%%%
\section{Conclusion}




%%%%%%%%%%%%%%%%%%%%%%%%%%
\section*{Acknowledgment}


%%%%%%%%%%%%%%%%%%%%%%%%%%
\bibliographystyle{IEEEtran}
\bibliography{dce-mri.bib}

\end{document}


