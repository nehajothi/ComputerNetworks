\documentclass[conference]{IEEEtran}

% for numbered citations
\usepackage{cite}
% for figures based on pdfs
\usepackage[pdftex]{graphicx}
 

\begin{document}
\title{Analyzing On Node Parallel Programming Models for Linear Solvers}


% author names and affiliations
% use a multiple column layout for up to three different
% affiliations
\author{\IEEEauthorblockN{Christoph Brehm}
\IEEEauthorblockA{Department of Aerospace and Mechanical Engineering\\
University of Arizona\\
Tucson, Arizona\\
Email: cbrehm@email.arizona.edu}
\and
\IEEEauthorblockN{Jacob Combs, Erman Gurses, Michelle Mills Strout}
\IEEEauthorblockA{Computer Science\\
University of Arizona\\
Tucson, Arizona\\
Email: \{jacobcombs, egurses, mstrout\}@email.arizona.edu}}

\maketitle

%%%%%%%%%%%%%%%%%%%%%%%%%%
\begin{abstract}
Computational fluid dynamics codes include solving a number of systems of linear equations.
Such codes are typically written in Fortran have been parallelized for distributed memory parallelism
using MPI.  However, MPI does not run directly on GPUs (FIXME: check this)
and does not scale well on shared memory machines.
In this paper we investigate the use of OpenMP, OpenCL, CUDA, and Chapel
as possible approaches to providing shared memory parallelism for the solving of linear
systems in these applications.
\end{abstract}

\IEEEpeerreviewmaketitle


%%%%%%%%%%%%%%%%%%%%%%%%%%
\section{Introduction}


Other researchers have investigated the performance of red-black successive over-relaxation 
in the parallel programming languages D, Go, and Chapel~\cite{Mittal2014}.
[FIXME: what did they conclude?]


Intro to GPU computing~\cite{Nickolls2010}

%%%%%%%%%%%%%%%%%%%%%%%%%%
\section{Questions for Someone to Answer}

\begin{itemize}
\item A number of solvers exist Hypre, PetsC, etc.   Why rewrote a solver for use within a particular application?
\end{itemize}

%%%%%%%%%%%%%%%%%%%%%%%%%%
\section{Conclusion}




%%%%%%%%%%%%%%%%%%%%%%%%%%
\section*{Acknowledgment}


%%%%%%%%%%%%%%%%%%%%%%%%%%
\bibliographystyle{IEEEtran}
\bibliography{linear-solve.bib}

\end{document}


