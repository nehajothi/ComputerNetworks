\documentclass[conference]{IEEEtran}

% for numbered citations
\usepackage{cite}
% for figures based on pdfs
\usepackage[pdftex]{graphicx}
 

\begin{document}
\title{Analyzing Parallel Programming Models for Analysis of Small Object Orbital Data}


% author names and affiliations
% use a multiple column layout for up to three different
% affiliations
\author{\IEEEauthorblockN{Kat Volk}
\IEEEauthorblockA{Lunary and Planetary Laboratory\\
University of Arizona\\
Tucson, Arizona\\
Email: http://katvolk.com}
\and
\IEEEauthorblockN{Benjamin James Gaska, Neha Jothi, Mahdi Soltan Mohammadi, \\and Michelle Mills Strout}
\IEEEauthorblockA{Computer Science\\
University of Arizona\\
Tucson, Arizona\\
Email: \{bengaska, nehaj, mahdi.s.m.k, mstrout\}@email.arizona.edu}}

\maketitle

%%%%%%%%%%%%%%%%%%%%%%%%%%
\begin{abstract}
When studying small objects in the Kuiper belt, a common approach is to simulate the orbits
of points that start near observed objects and then analyze the orbits of those points to determine
characteristics about them.  One key piece of information is whether the object is in orbital resonance
with Neptune.  Currently this information is discovered through analysis codes written
in Perl.  The problem is that the analysis experiences significant load imbalance and interpretation
overhead, both of which result in the analysis of the time consuming objects requiring
about 15 minutes.  With the new LLSST telescope that will be able
to survey approximately 10K objects within a year or two, the need to do this analysis
more efficiently is critical.
At 5 minutes for the time consuming cases the necessary once a month analyses would
 take over 3 months.
In this paper, we improve the execution time of the analysis by more than an order of 
magnitude.  We do this by porting the analysis to various programming languages:
parallel perl, Python, Chapel, Go, OpenMP, MPI, and Cilk++.
We compare these implementations in terms of performance, accuracy,
and maintainability from the perspective of the scientists.  We also
identify the key problems in this workload in terms of parallelization and
experiment with some possible parallelizations that help with the 
significant load imbalance.
\end{abstract}

\IEEEpeerreviewmaketitle

%%%%%%%%%%%%%%%%%%%%%%%%%%
\section{Questions}

\begin{itemize}
\item Who else is writing analysis codes for discovering orbital resonance and libration and
what programming language are they using?
People describe what analyses they are doing but not how they are coding it up and 
software not typically shared. Publish results of such analyses.
Typically Fortran is what is posted.
Observational survey posting Fortran and Python code to github.
Saw these objects, what are they doing?  Code for classifying might not be publically
available due to support reasons.

\item How much variance is there between objects that take require checking most of the
ratios and those that require few checks?  What are some concrete numbers?
Kat might be able to give us a distribution on this. (60\% ish)

\item Can put into models about history of Neptune.

\item What does it mean when we say that the classification has happened manually?
Every single classification is still checked by an eye.  Could do a second check?

\item Year that LSST should come online?  5 to 10 years from now.

\item Do we only need to do the analysis for each object once?
Everytime fit the orbit with new observations.  Whole sky once per month.
\end{itemize}

%%%%%%%%%%%%%%%%%%%%%%%%%%
\section{Introduction}

~\cite{ChapelOverviewJan13}

Astronomers and planetary scientists would like to know which real
objects in the Kuiper belt are resonant with Neptune.
[Why do we want to know this?]

Ideally this classification would be done automatically by 
extrapolating the orbits of observed objects and then analyzing
the simulated orbits to determine if resonance with Neptune occurs
and if so which resonance.  Currently this process is done manually
[what does this mean exactly?] with approximately
2000 objects in the last 10 years being classified in
groups of 10 to 20~\cite{ClassificationCite}.
When the Large Synoptic Survey Telescope (LSST) comes online in  year(?)
the expectation is that ten thousand objects
will be observed within a year or two.

Analyses are being programmed in Perl and ...  The problem
is that these analyses exhibit a significant amount of load imbalance
and the peak analysis time for an individual object is 
on the order of 45 minutes(?).
[Math showing that this will not scale.]

Argument for speed
\begin{verbatim}
10000*5: 50000   ! number of objects times number of minutes for a long one. 
#/60: 833.333333 
#/24: 34.722222  ! number of days
would really like to do this on 3 time scales, short, medium, and long term
might want more precision with more stuff in time files
\end{verbatim}

%%%%%%%%%%%%%%%%%%%%%%%%%%
\section{Methodology}

~\cite{ChapelOverviewJan13}


%%%%%%%%%%%%%%%%%%%%%%%%%%
\section{Conclusion}




%%%%%%%%%%%%%%%%%%%%%%%%%%
\section*{Acknowledgment}


%%%%%%%%%%%%%%%%%%%%%%%%%%
\bibliographystyle{IEEEtran}
\bibliography{libration.bib}

\end{document}


