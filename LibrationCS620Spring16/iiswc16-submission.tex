\documentclass[conference]{IEEEtran}

% for numbered citations
\usepackage{cite}
% for figures based on pdfs
\usepackage[pdftex]{graphicx}
 

\begin{document}
\title{Analyzing Parallel Programming Models for Analysis of Small Object Orbital Data}


% author names and affiliations
% use a multiple column layout for up to three different
% affiliations
\author{\IEEEauthorblockN{Kat Volk}
\IEEEauthorblockA{Lunary and Planetary Laboratory\\
University of Arizona\\
Tucson, Arizona\\
Email: http://katvolk.com}
\and
\IEEEauthorblockN{Benjamin James Gaska, Neha Jothi, Mahdi Soltan Mohammadi, \\and Michelle Mills Strout}
\IEEEauthorblockA{Computer Science\\
University of Arizona\\
Tucson, Arizona\\
Email: \{bengaska, nehaj, mahdi.s.m.k, mstrout\}@email.arizona.edu}}

\maketitle

%%%%%%%%%%%%%%%%%%%%%%%%%%
\begin{abstract}
When studying small objects in the Kuiper belt, a common approach is to simulate the orbits
of points that start near observed objects and then analyze the orbits of those points to determine
characteristics about them.  One key piece of information is whether the object is in orbital resonance
with Neptune.  Currently this information is discovered through analysis codes written
in Perl.  The problem is that the analysis experiences significant load imbalance and interpretation
overhead, both of which result in analyzing around 100 objects taking ?? minutes
on a ???(probably HPC machine for this).  With the new LLSST telescope that will be able
to survey approximately 10K objects within a year or two, the need to do this analysis
more efficiently is critical.
In this case study, the original implementation to analyze the simulation data was
written in Perl.  We compare this implementation in terms of performance and accuracy
to implementations in parallel perl, Python, Chapel, Go, OpenMP, ...
We find ...
\end{abstract}

\IEEEpeerreviewmaketitle


%%%%%%%%%%%%%%%%%%%%%%%%%%
\section{Introduction}

~\cite{ChapelOverviewJan13}

%%%%%%%%%%%%%%%%%%%%%%%%%%
\section{Conclusion}




%%%%%%%%%%%%%%%%%%%%%%%%%%
\section*{Acknowledgment}


%%%%%%%%%%%%%%%%%%%%%%%%%%
\bibliographystyle{IEEEtran}
\bibliography{libration.bib}

\end{document}


